\documentclass[draft]{article}

% - Style
\usepackage{base}

% - Title
\gdef\theassessment{PHYS4001 - Literature Review}
\gdef\thesupervisor{Professor Igor Bray}
\gdef\theinstitution{Curtin University}
\gdef\thestudentid{1834 2884}

\title{Ionisation Amplitudes in Electron-Impact Helium Collisions within the
  S-Wave Model}
\author{Tom Ross}
\date{\today}

% - Headers
\pagestyle{fancy}
\fancyhf{}
\rhead{\theauthor}
\chead{}
\lhead{\theassessment}
\rfoot{\thepage}
\cfoot{}
\lfoot{}

%
\setcounter{secnumdepth}{3}

% document

\begin{document}

% cover page

\begin{titlepage}
  \begin{flushleft}
    \theinstitution \hfill \theassessment
  \end{flushleft}
  \hrule
  \begin{center}
    {
      \huge
      \thetitle
    }
    ~\\
    \rule[1.0pt]{8.5cm}{0.4pt}
    ~\\
    {
      \large
      \theauthor ~supervised by \thesupervisor
    }
  \end{center}
  \hrule
  \begin{center}
    [ABSTRACT]
  \end{center}
\end{titlepage}

\clearpage

% contents page

\tableofcontents

\listoffigures

\listoftables

% statement

\clearpage

\section{Introduction}
\label{sec:introduction}

\paragraph{Applications of Electron-Impact Hydrogen Scattering}

\paragraph{Specific Applications of Electron-Impact Hydrogen Ionisation}

\paragraph{Development of Quantum Scattering Theory}

\section{Theory}
\label{sec:theory}

\todo{Remove unnecessary commas.}

\todo{Replace 'expansion' with 'representation'.}

We shall describe the development of the Convergent Close-Coupling (CCC) method
for generalised projectile-target scattering, before describing its application
to the cases of: electron-impact hydrogen (e-H) scattering, and electron-impact
helium (e-He) scattering.
In particular, we shall explore the treatment of target ionisation within the CCC
method.
Note that in the general treatment, we restrict our attention to electron
projectiles and atomic/ionic targets consisting of $n_{\mathrm{e}}$ electrons.

\subsection{Convergent Close-Coupling Method}
\label{sec:ccc-method}

In brief, the CCC method utilises the method of basis expansion, discussed in
further detail in \autoref{app:basis-expansion}, to numerically solve the
Lippmann-Schwinger equation, for a projectile-target system, to yield the
transition amplitudes, which are convergent as the size of the basis is
increased.
The rate of convergence depends on many factors, such as the complexity and
geometry of the projectile-target system for example, as well as the choice of
basis used in the expansion.
Furthermore, by selecting a complete basis, ionisation transition amplitudes can
be treated in a similar manner to discrete excitation transition amplitudes.

\subsubsection{Laguerre Basis}
\label{sec:laguerre-basis}

To describe the target structure, the CCC method utilises a Laguerre basis,
$\lrset{\ket{\varphi_{i}}}_{i = 1}^{\infty}$, for the Hilbert space
$L^{2}\lr{\real^{3}}$, for which the coordinate-space representation is of the
form
\begin{equation}
  \label{eq:laguerre-basis}
  \bra{\boldsymbol{r}}
  \ket{\varphi_{i}}
  =
  \varphi_{i}\lr{r, \Omega}
  =
  \tfrac{1}{r}
  \xi_{k_{i}, l{i}}\lr{r}
  Y_{l_{i}}^{m_{i}}\lr{\Omega}
  ,
\end{equation}
where $Y_{l_{i}}^{m_{i}}\lr{\Omega}$ are the spherical harmonics, and where
$\xi_{k_{i}, l_{i}}\lr{r}$ are the Laguerre radial basis functions, which are of
the form
\begin{equation}
  \label{eq:laguerre-radial-basis}
  \xi_{k, l}\lr{r}
  =
  \sqrt
  {
    \dfrac
    {
      \lambda_{l}
      \lr{k - 1}!
    }
    {
      \lr{2l + 1 + k}!
    }
  }
  \lr{\lambda_{l} r}^{l + 1}
  \exponential\lr[\big]{- \tfrac{1}{2} \lambda_{l} r}
  L_{k - 1}^{2l + 2}\lr{\lambda_{l} r}
  ,
\end{equation}
where $\lambda_{l}$ is the exponential fall-off, for each $l$, and where
$L_{k - 1}^{2l + 2}\lr{\lambda_{l} r}$ are the associated Laguerre polynomials.
Note that we must have that
$k_{i} \in \lrset{1, 2, \dotsc}$,
$l_{i} \in \lrset{0, 1, \dotsc}$ and
$m_{i} \in \lrset{-\ell_{i}, \dotsc, \ell_{i}}$, for each
$i \in \lrset{1, 2, \dotsc}$.

This Laguerre basis is utilised due to: the Laguerre basis functions,
$\lrset{\varphi_{i}\lr{r, \Omega}}_{i = 1}^{\infty}$, forming a complete basis
for the Hilbert space $L^{2}\lr{\real^{3}}$ - shown in
\autoref{app:laguerre-completeness}, the short-range and long-range behaviour of
the radial basis functions being well suited to describing bound target states
and providing a basis for expanding continuum states in, and because it allows
the matrix elements of certain operators to be calculated analytically.

Practically, we cannot utilise a basis of infinite size.
Hence, we truncate the Laguerre radial basis,
$\lrset{\xi_{k, l}\lr{r}}_{k = 1}^{N_{l}}$, to a certain number of radial basis
functions, $N_{l}$, for each $l$, and we also truncate
$l \in \lrset{0, \dotsc, l_{\max}}$,
limiting the maximum angular momentum we consider in our basis.
Hence, for a given value of $m$, we have a basis size of
\begin{equation}
  \label{eq:basis-size}
  N
  =
  \sum_{l = 0}^{l_{\max}}
  N_{l}
  .
\end{equation}
In the limit as $N \to \infty$, the truncated basis will tend towards
completeness, and it is in this limit that we discuss the convergence of the
Convergent Close-Coupling method.

Further properties of the Laguerre basis are discussed in
\autoref{app:laguerre-basis}.

\subsubsection{Target States}
\label{sec:target-states}

Possessing now a suitable basis to work with, we proceed to represent the
target in this basis by the method of basis expansion.
Firstly, we note that electrons are indistinguishable fermionic particles; that
is, no two electrons can be distinguished from each other, and they must satisfy
Pauli's exclusion principle - that an electron state cannot be occupied by more
than one electron.
Since electrons are indistinguishable, we might naively suppose that the space
of states consisting of $n$ electrons is simply the $n$-th tensor power of the
one-electron space, $T^{n}\lr{\mathcal{H}}$, defined by
\begin{equation}
  \label{eq:tensor-power}
  T^{n}\lr{\mathcal{H}}
  =
  \lrset
  {
    \ket{\psi_{1}}
    \otimes
    \dots
    \otimes
    \ket{\psi_{n}}
    :
    \ket{\psi_{1}},
    \dotsc,
    \ket{\psi_{n}}
    \in \mathcal{H}
  }
  ,
\end{equation}
where $\mathcal{H}$ is the space of one-electron states.
However this fails to account for Pauli's exclusion principle, since any
one-electron state may be occupied up to $n$ times.
Hence, the space of states consisting of $n$ electrons is instead defined to be
the quotient space $\Lambda^{n}\lr{\mathcal{H}}$ of $T^{n}\lr{\mathcal{H}}$ by
$\mathcal{D}^{n}$,
\begin{equation}
  \label{eq:exterior-power}
  \Lambda^{n}\lr{\mathcal{H}}
  =
  T^{n}\lr{\mathcal{H}}
  /
  \mathcal{D}^{n}
  ,
\end{equation}
where $\mathcal{D}^{n} \subset T^{n}\lr{\mathcal{H}}$ is the subspace of tensor
products which contain any one-electron state more than once.
The space $\Lambda^{n}\lr{\mathcal{H}}$ is known as the $n$-th exterior power of
$\mathcal{H}$, and is identifiable as the subspace of $T^{n}\lr{\mathcal{H}}$
consisting of anti-symmetric tensors.

It follows that for an atomic/ionic target, consisting of $n_{\mathrm{e}}$
electrons, the space of target states is of the form
$\mathcal{H}_{T} = \Lambda^{n_{\mathrm{e}}}\lr{\mathcal{H}}$.
Note that we shall adopt the convention that the projectile electron space is
denoted by $\mathcal{H}_{0}$, and the $m$-th target electron space by
$\mathcal{H}_{m}$, for $m = 1, \dotsc, n_{\mathrm{e}}$.
Furthermore, operators which act on the $m$-th electron space (including the
projectile electron), will be indexed by $m$, for
$m = 0, 1, \dotsc, n_{\mathrm{e}}$.

\paragraph{Target Hamiltonian}
\label{sec:target-hamiltonian}

The target Hamiltonian, for an atomic/ionic target with $n_{\mathrm{e}}$
electrons, is of the form
\begin{equation}
  \label{eq:target-hamiltonian}
  \hat{H}_{T}
  =
  \sum_{m = 1}^{n_{\mathrm{e}}}
  \hat{K}_{m}
  +
  \sum_{m = 1}^{n_{\mathrm{e}}}
  \hat{V}_{m}
  +
  \sum_{m = 1}^{n_{\mathrm{e}}}
  \sum_{n = m + 1}^{n_{\mathrm{e}}}
  \hat{V}_{m, n}
  ,
\end{equation}
where $\hat{K}_{m}$ and $\hat{V}_{m}$ are the target electron kinetic and
electron-nuclei potential operators, for $m = 1, \dotsc, n_{\mathrm{e}}$, and
where $\hat{V}_{m, n}$ are the electron-electron potential operators, for
$m, n = 1, \dotsc, n_{\mathrm{e}}$.

\paragraph{Target Diagonalisation}
\label{sec:target-diagonalisation}

The target Hamiltonian, restricted to just one target electron,
\begin{equation}
  \label{eq:target-1e-hamiltonian}
  \hat{H}_{T, \mathrm{e}}^{}
  =
  \hat{K}_{1}
  +
  \hat{V}_{1}
  ,
\end{equation}
is expanded in a Laguerre basis, $\lrset{\ket{\varphi_{i}}}_{i = 1}^{N}$, and
diagonalised to yield a set of one-electron atomic orbitals
$\lrset{\ket*{\phi_{i}^{\lr{N}}}}_{i = 1}^{N}$, which are orthonormal and
satisfy
\begin{equation}
  \label{eq:target-1e-orbitals}
  \bra*{\phi_{i}^{\lr{N}}}
  \hat{H}_{T, \mathrm{e}}
  \ket*{\phi_{j}^{\lr{N}}}
  =
  \varepsilon_{i}^{\lr{N}}
  \delta_{i, j}
  .
\end{equation}
From these one-electron atomic orbitals, we generate a set of one-electron spin
orbitals, $\lrset{\ket*{\chi_{i}^{\lr{N}}}}_{i = 1}^{2N}$, for which
$\ket*{\chi_{2i - 1}^{\lr{N}}}$ and $\ket*{\chi_{2i}^{\lr{N}}}$ both correspond
to $\ket*{\phi_{i}^{\lr{N}}}$ but have spin projection $\tfrac{1}{2}$ and
$-\tfrac{1}{2}$ respectively.
These one-electron spin orbitals are then combined to construct Slater
determinants; for any selection of $n_{\mathrm{e}}$ one-electron spin orbitals
$\ket*{\chi_{a_{1}}^{\lr{N}}}, \dotsc, \ket*{\chi_{a_{n_{\mathrm{e}}}}^{\lr{N}}}
\in \lrset{\ket*{\chi_{i}^{\lr{N}}}}_{i = 1}^{2N}$, the Slater determinant of
these spin orbitals is of the form
\begin{equation}
  \label{eq:target-slater-determinant}
  \ket*
  {
    \chi_{a_{1}}^{\lr{N}},
    \dotsc,
    \chi_{a_{n_{\mathrm{e}}}}^{\lr{N}}
  }
  =
  \dfrac{1}{\sqrt{n_{\mathrm{e}}!}}
  \sum_{\sigma \in S_{n_{\mathrm{e}}}}
  \mathrm{sgn}\lr{\sigma}
  \ket*{\chi_{a_{\sigma\lr{1}}}^{\lr{N}}}
  \otimes
  \dots
  \otimes
  \ket*{\chi_{a_{\sigma\lr{n_{\mathrm{e}}}}}^{\lr{N}}}
  \in
  \mathcal{H}_{T}
  ,
\end{equation}
where $S_{n_{\mathrm{e}}}$ is the symmetric group on $n_{\mathrm{e}}$ elements,
the sum is taken over all permutations, $\sigma \in S_{n_{\mathrm{e}}}$, and
where $\mathrm{sgn}\lr{\sigma}$ is the signature of the permutation $\sigma$.
We note that Slater determinants are anti-symmetric under pairwise exchange of
any two orbitals, and are zero if constructed with two spin orbitals in the same
state, and hence adhere to Pauli's exclusion principle and are indeed elements
of $\mathcal{H}_{T} = \Lambda^{n_{\mathrm{e}}}\lr{\mathcal{H}}$.

The true target states,
$\lrset{\ket{\Phi_{i}}}_{i = 1}^{\infty} \in \mathcal{H}_{T}$, are then
approximated by expanding the many-electron target Hamiltonian, $\hat{H}_{T}$,
in a basis of Slater determinants,
\begin{equation}
  \label{eq:target-slater-determinant-basis}
  \lrset[\big]
  {
    \ket*
    {
      \chi_{a_{1}}^{\lr{N}},
      \dotsc,
      \chi_{a_{n_{\mathrm{e}}}}^{\lr{N}}
    }
    :
    a_{1}, \dotsc, a_{n_{\mathrm{e}}}
    \in
    \lrset{1, \dotsc, 2N}
  }
  ,
\end{equation}
and diagonalising to yield a set of target pseudostates,
$\lrset{\ket*{\Phi_{i}^{\lr{N}}}}_{i = 1}^{N_{T}}$, which are orthonormal and
satisfy
\begin{equation}
  \label{eq:target-states}
  \bra*{\Phi_{i}^{\lr{N}}}
  \hat{H}_{T}
  \ket*{\Phi_{j}^{\lr{N}}}
  =
  \epsilon_{i}^{\lr{N}}
  \delta_{i, j}
  .
\end{equation}
Note that the number of target pseudostates, $N_{T}$, depends on the number of
Slater determinants utilised in the expansion of $\hat{H}_{T}$.
The process of selecting which Slater determinants to use in the expansion is
not trivial, and will be discussed in further detail in
\autoref{sec:he-frozen-core}.

Note also that the $\lr{N}$ superscript has been introduced to indicate that
these are not true eigenstates of the target Hamiltonian, only of its
representation in the truncated Laguerrre basis, and that these pseudostates and
their pseudoenergies are dependent on the size of the Laguerre basis utilised.
The diagonalisation procedure is discussed in further detail in
\autoref{app:diagonalisation}.

\paragraph{Completeness of Target Pseudostates}
\label{sec:target-states-completeness}

As a result of the completeness of the Laguerre basis, the set of target
pseudostates will be separable into a set of bounded pseudostates which will
form an approximation of the true target discrete spectrum, and a set of
unbounded pseudostates which will provide a discretisation of the true continuum
of unbounded states.
We order the target pseudostates by increasing pseudoenergy,
$\epsilon_{1}^{\lr{N}} < \dotsc < \epsilon_{N_{T}}^{\lr{N}}$, which allows us to
express the separability of the spectrum in the form
\begin{equation}
  \label{eq:target-spectrum-separable}
  \lrset{\ket*{\Phi_{i}^{\lr{N}}}}_{i = 1}^{N_{T}}
  =
  \lrset{\ket*{\Phi_{i}^{\lr{N}}}}_{i = 1}^{N_{B}}
  \cup
  \lrset{\ket*{\Phi_{i}^{\lr{N}}}}_{i = N_{B} + 1}^{N_{T}}
  ,
\end{equation}
where $\epsilon_{i}^{\lr{N}} < 0$ for $i = 1, \dotsc, N_{B}$, and where
$\epsilon_{i}^{\lr{N}} \geq 0$ for $i = N_{B} + 1, \dotsc, N_{T}$.
Note that $N_{B}$ is the number of bounded pseudostates, and we write
$N_{U} = N_{T} - N_{B}$ to represent the number of unbounded pseudostates, both of
which are dependent on $N$ by consequence of the construction of the target
pseudostates.

We note that the projection operator for the target pseudostates,
$\hat{I}_{T}^{\lr{N}}$, is of the form
\begin{equation}
  \label{eq:target-projection}
  \hat{I}_{T}^{\lr{N}}
  =
  \sum_{i = 1}^{N_{T}}
  \ket*{\Phi_{i}^{\lr{N}}}
  \bra*{\Phi_{i}^{\lr{N}}}
  =
  \sum_{i = 1}^{N_{B}}
  \ket*{\Phi_{i}^{\lr{N}}}
  \bra*{\Phi_{i}^{\lr{N}}}
  +
  \sum_{i = N_{B} + 1}^{N_{T}}
  \ket*{\Phi_{i}^{\lr{N}}}
  \bra*{\Phi_{i}^{\lr{N}}}
  ,
\end{equation}
and so in the limit as $N \to \infty$, the sum over the bounded pseudostates
will converge to the sum over the true target discrete states
and the sum over the unbounded pseudostates will converge to a discretisation of
the integral over the true continuum spectrum.
Whence, it follows that projection operator for the target pseudostates
converges to the identity operator, for $\mathcal{H}_{T}$,
in the limit as $N \to \infty$; that is,
\begin{equation}
  \label{eq:target-projection-convergence}
  \lim_{N \to \infty}
  \hat{I}_{T}^{\lr{N}}
  =
  \hat{I}_{T}
  .
\end{equation}

\subsubsection{Projectile States}
\label{sec:projectile-states}

The projectile states, $\ket{\boldsymbol{k}} \in \mathcal{H}$, are defined to be
eigenstates of the free Hamiltonian; that is,
\begin{equation}
  \label{eq:projectile-hamiltonian}
  \hat{K}_{0}
  \ket{\boldsymbol{k}}
  =
  \tfrac{k^{2}}{2}
  \ket{\boldsymbol{k}}
  .
\end{equation}
It follows that the coordinate-space representation of projectile states are
plane waves; that is,
\begin{equation}
  \label{eq:projectile-states}
  \bra{\boldsymbol{r}}
  \ket{\boldsymbol{k}}
  =
  \lr{2 \pi}^{-\tfrac{3}{2}}
  \exponential\lr
  {
    \imath
    \boldsymbol{k}
    \vdot
    \boldsymbol{r}
  }
  .
\end{equation}
Further properties of the plane waves are discussed in
\autoref{app:plane-waves}.

\subsubsection{Total Wavefunction}
\label{sec:total-wavefunction}

The total wavefunction
$\ket*{\Psi^{\lr{+}}} \in \Lambda^{1 + n_{\mathrm{e}}}\lr{\mathcal{H}}$ is
defined to be an eigenstate of the total Hamiltonian $\hat{H}$ with total
energy $E$ and specified to have outgoing spherical-wave boundary conditions,
\begin{equation}
  \label{eq:total-state}
  \hat{H}
  \ket*{\Psi^{\lr{+}}}
  =
  E
  \ket*{\Psi^{\lr{+}}}
  ,
\end{equation}
where $\hat{H}$ is of the form
\begin{equation}
  \label{eq:total-hamiltonian}
  \hat{H}
  =
  \hat{H}_{T}
  +
  \hat{K}_{0}
  +
  \hat{V}_{0}
  +
  \sum_{m = 1}^{n_{\mathrm{e}}}
  \hat{V}_{0, m}
  ,
\end{equation}
where $\hat{H}_{T}$ is the target Hamiltonian, defined in
\autoref{eq:target-hamiltonian}, $\hat{K}_{0}$ is the projectile electron
kinetic operator, $\hat{V}_{0}$ is the projectile electron-nuclei potential
operator, and $\hat{V}_{0, m}$ are the projectile electron-target electron
potential operators.

To ensure that the total state is anti-symmetric, we construct it using a
multichannel expansion of the form
\begin{equation}
  \label{eq:total-state-multichannel}
  \ket*{\Psi^{\lr{+}}}
  =
  \lrsq[\bigg]
  {
    1
    -
    \sum_{m = 1}^{n_{\mathrm{e}}}
    \hat{P}_{0, m}
  }
  \ket*{\psi^{\lr{+}}}
\end{equation}
where $\hat{P}_{0, m}$ are the pairwise electron exchange operators, for
$m = 1, \dotsc, n_{\mathrm{e}}$, exchanging the projectile electron state and
$m$-th target electron state, and where
$\ket*{\psi^{\lr{+}}} \in
\bigotimes_{m = 1}^{1 + n_{\mathrm{e}}} \mathcal{H}_{\mathrm{e}}$ is the
un-symmetrised total state.
We neglect to anti-symmetrise the target electrons with each other, since states
in the target space are already anti-symmetric by construction.

To construct the un-symmetrised total state, $\ket*{\psi^{\lr{+}}}$, we
observe, as a result of \autoref{eq:target-projection-convergence}, that
\begin{equation}
  \label{eq:total-state-un-symmetrised-convergence}
  \ket*{\psi^{\lr{+}}}
  =
  \lim_{N \to \infty}
  \hat{I}_{T}^{\lr{N}}
  \ket*{\psi^{\lr{+}}}
  =
  \lim_{N \to \infty}
  \ket*{\psi^{\lr{N, +}}}
\end{equation}
where we have defined
\begin{equation}
  \label{eq:total-state-un-symmetrised-n}
  \ket*{\psi^{\lr{N, +}}}
  =
  \hat{I}_{T}^{\lr{N}}
  \ket*{\psi^{\lr{+}}}
  =
  \sum_{i = 1}^{N_{T}}
  \ket*{\Phi_{i}^{\lr{N, +}}}
  \bra*{\Phi_{i}^{\lr{N, +}}}
  \ket*{\psi^{\lr{+}}}
  =
  \sum_{i = 1}^{N_{T}}
  \ket*{\Phi_{i}^{\lr{N, +}}}
  \otimes
  \ket*{F_{i}^{\lr{N, +}}}
\end{equation}
where $\ket*{F_{i}^{\lr{N, +}}}
= \bra*{\Phi_{i}^{\lr{N, +}}}\ket*{\psi^{\lr{+}}}
\in \mathcal{H}_{\mathrm{e}}$.
Similarly, we observe that
\begin{equation}
  \label{eq:total-state-multichannel-convergence}
  \ket*{\Psi^{\lr{+}}}
  =
  \lim_{N \to \infty}
  \ket*{\Psi^{\lr{N, +}}}
\end{equation}
where we have defined
\begin{equation}
  \label{eq:total-state-multichannel-n}
  \ket*{\Psi^{\lr{N, +}}}
  =
  \lrsq[\bigg]
  {
    1
    -
    \sum_{m = 1}^{n_{\mathrm{e}}}
    \hat{P}_{0, m}
  }
  \ket*{\psi^{\lr{N, +}}}
  .
\end{equation}

However, after projecting the un-symmetrised total state with the projection
operator for the target pseudostates, the multichannel expansion is not uniquely
defined, since for any state, $\ket*{\omega^{\lr{N, +}}}$, such that
\begin{equation}
  \label{eq:multichannel-kernel}
  \ket*{\omega^{\lr{N, +}}}
  \in
  \ker\lr[\bigg]
  {
    \lrsq[\bigg]
    {
      1 - \sum_{m = 1}^{n_{\mathrm{e}}}\hat{P}_{0, m}
    }
    \hat{I}_{T}^{\lr{N}}
  }
\end{equation}
and constant $\theta$, the multichannel expansion of
$\ket*{\psi^{\lr{N, +}}}
+ \theta \hat{I}_{T}^{\lr{N}} \ket*{\omega^{\lr{N, +}}}$ will be identical to
that of $\ket*{\psi^{\lr{N, +}}}$.
To resolve this dilemma, we impose the constraint that for any of the
one-electron atomic orbitals,
$\ket*{\phi_{j}^{\lr{N}}} \in \mathcal{H}_{\mathrm{e}}$, used in the
construction of the target pseudostates in \autoref{eq:target-1e-hamiltonian},
that
\begin{equation}
  \label{eq:multichannel-constraint}
  \bra*{\phi_{j}^{\lr{N}}}
  \otimes
  \bra*{\Phi_{i}^{\lr{N}}}
  \hat{P}_{0, m}
  \ket*{\Gamma^{\lr{N, +}}}
  =
  -
  \bra*{\phi_{j}^{\lr{N}}}
  \otimes
  \bra*{\Phi_{i}^{\lr{N}}}
  \ket*{\Gamma^{\lr{N, +}}}
\end{equation}
for all $\ket*{\Gamma^{\lr{N, +}}}
\in \bigotimes_{m = 1}^{1 + n_{\mathrm{e}}} \mathcal{H}_{\mathrm{e}}$.
\todo[multichannel constraint]{Is the ket in this constraint correct?}
Whence it follows that the multichannel expansion of
$\ket*{\psi^{\lr{N, +}}}
+ \theta \hat{I}_{T}^{\lr{N}} \ket*{\omega^{\lr{N, +}}}$ will be identical to
that of $\ket*{\psi^{\lr{N, +}}}$ only in the case where $\theta = 0$, or
$\ket*{\omega^{\lr{N, +}}} = \ket*{0}$; that is to say,
$\ket*{\psi^{\lr{N, +}}}$ will now uniquely determine $\ket*{\Psi^{\lr{N, +}}}$.

\subsubsection{Close-Coupling Equations}
\label{sec:cc-equations}

\subsubsection{Transition Amplitudes}
\label{sec:transition-amplitudes}

\subsubsection{Cross Sections}
\label{sec:cross-sections}

\paragraph{Total Cross Sections}
\label{sec:cc-total-cross-sections}

\paragraph{Differential Cross Sections}
\label{sec:cc-differential-cross-sections}

\subsubsection{S-Wave Model}
\label{sec:s-wave-model}

\subsection{Electron-Impact Hydrogen Scattering}
\label{sec:e-h}

\subsubsection{Elastic Scattering}
\label{sec:e-h-elastic-scattering}

\subsubsection{Excitation}
\label{sec:e-h-excitation}

\subsubsection{Ionisation}
\label{sec:e-h-ionisation}

\paragraph{Singlet Case}
\label{sec:e-h-singlet}

\paragraph{Triplet Case}
\label{sec:e-h-triplet}

\subsection{Electron-Impact Helium Scattering}
\label{sec:e-he}

\subsubsection{Additional Considerations for a Helium Target}
\label{sec:he-target}

\paragraph{Frozen-Core Model}
\label{sec:he-frozen-core}

\paragraph{Meta-stable States}
\label{sec:he-meta-stable}

\todo[meta-stable states]{
  How are positive energy discrete states handled?
  Do positive energy discrete states overlap with the continuum?
  Do we simply include the positive energy discrete states in the continuum with
  a Dirac mass function?
}

\subsubsection{Elastic Scattering}
\label{sec:e-he-elastic-scattering}

\subsubsection{Excitation}
\label{sec:e-he-excitation}

\paragraph{Auto-Ionisation}
\label{sec:e-he-auto-ionisation}

\subsubsection{Ionisation}
\label{sec:e-he-ionisation}

\section{Survey of Experimental Literature}
\label{sec:survey-experimental}

\section{Survey of Theoretical Literature}
\label{sec:survey-theoretical}

\subsection{Electron-Impact Hydrogen Ionisation Calculations}
\label{sec:e-h-ionisation-calculations}

\subsubsection{Convergent Close-Coupling Calculations}
\label{sec:e-h-ccc-calculations}

\subsubsection{Exterior-Complex-Scaling Calculations}
\label{sec:e-h-ecs-calculations}

\subsubsection{Ansatz of Zatsarinny and Bartschat}
\label{sec:e-h-ecs-calculations}

\subsection{Electron-Impact Helium Ionisation Calculations}
\label{sec:e-he-ionisation-calculations}

\subsubsection{Convergent Close-Coupling Calculations}
\label{sec:e-he-ccc-calculations}

\subsubsection{Exterior-Complex-Scaling Calculations}
\label{sec:e-he-ecs-calculations}

\subsubsection{Ansatz of Zatsarinny and Bartschat}
\label{sec:e-he-ecs-calculations}

\section{Conclusion}
\label{sec:conclusion}

\clearpage

\bibliographystyle{chicago}

\bibliography{references}

\clearpage

\appendix

\section{Properties of Utilised Bases}
\label{app:properties}

\subsection{Spherical Harmonics}
\label{app:spherical-harmonics}

\subsubsection{Completeness}
\label{app:spherical-harmonic-completeness}

\todo[spherical harmonic completeness]{
  Prove that the set of spherical harmonics forms,
  $\lrset{Y_{l}^{-l}\lr{\Omega}, \dotsc, Y_{l}^{l}\lr{\Omega}}_{l = 0}^{\infty}$,
  forms an orthonormal, complete basis for the Hilbert space
  $L^{2}\lr{S^{2}}$.
}

\subsection{Laguerre Radial Basis}
\label{app:laguerre-radial-basis}

\subsubsection{Completeness}
\label{app:laguerre-radial-completeness}

\todo[laguerre radial completeness]{
  Prove that the Laguerre radial basis functions,
  $\lrset{\xi_{k, l}\lr{r}}_{k = 1}^{\infty}$, for each $l$, forms a complete
  basis for the Hilbert space $L^{2}\lr{[0, \infty)}$.
}

\subsection{Laguerre Basis}
\label{app:laguerre-basis}

\subsubsection{Completeness}
\label{app:laguerre-completeness}

It is shown in \autoref{app:laguerre-radial-completeness}, that the Laguerre
radial basis functions, $\lrset{\xi_{k, l}\lr{r}}_{k = 1}^{\infty}$, for each
$l$, forms a complete basis for the Hilbert space $L^{2}\lr{[0, \infty)}$.
Similarly, it is also shown in \autoref{app:spherical-harmonic-completeness},
that the set of spherical harmonics,
$\lrset{Y_{l}^{-l}\lr{\Omega}, \dotsc, Y_{l}^{l}\lr{\Omega}}_{l = 0}^{\infty}$,
forms an orthonormal, complete basis for the Hilbert space $L^{2}\lr{S^{2}}$.
Hence, the Laguerre basis functions
$\lrset{\varphi_{i}\lr{r, \Omega}}_{i = 1}^{\infty}$, form a complete basis
for the Hilbert space $L^{2}\lr{\real^{3}}$.

\subsection{Plane Waves}
\label{app:plane-waves}

\section{Numerical Techniques}
\label{app:numerical-techniques}

\subsection{Basis Expansion}
\label{app:basis-expansion}

\todo[basis expansion]{
  Elaborate on the method of basis expansion for Hilbert spaces.
}

\subsection{Diagonalisation}
\label{app:diagonalisation}

\todo[target diagonalisation]{
  Elaborate on the diagonalisation procedure for the target states.
}

\clearpage

\todos

\end{document}