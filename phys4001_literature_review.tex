\documentclass[draft]{article}

% - Style
\usepackage{base}

% - Title
\gdef\theassessment{PHYS4001 - Literature Review}
\gdef\thesupervisor{Professor Igor Bray}
\gdef\theinstitution{Curtin University}
\gdef\thestudentid{1834 2884}

\title{Ionisation Amplitudes in Electron-Impact Helium Collisions within the
  S-Wave Model}
\author{Tom Ross}
\date{\today}

% - Headers
\pagestyle{fancy}
\fancyhf{}
\rhead{\theauthor}
\chead{}
\lhead{\theassessment}
\rfoot{\thepage}
\cfoot{}
\lfoot{}

% document

\begin{document}

% cover page

\begin{titlepage}
  \begin{flushleft}
    \theinstitution \hfill \theassessment
  \end{flushleft}
  \hrule
  \begin{center}
    {
      \huge
      \thetitle
    }
    ~\\
    \rule[1.0pt]{8.5cm}{0.4pt}
    ~\\
    {
      \large
      \theauthor ~supervised by \thesupervisor
    }
  \end{center}
  \hrule
  \begin{center}
    [ABSTRACT]
  \end{center}
\end{titlepage}

\clearpage

% contents page

\tableofcontents

\listoffigures

\listoftables

% statement

\clearpage

\section{Introduction}
\label{sec:introduction}

\paragraph{Applications of Electron-Impact Hydrogen Scattering}

\paragraph{Specific Applications of Electron-Impact Hydrogen Ionisation}

\paragraph{Development of Quantum Scattering Theory}

\section{Theory}
\label{sec:theory}

\todo{Remove unnecessary commas.}

\todo{Replace 'expansion' with 'representation'.}

We shall describe the development of the Convergent Close-Coupling (CCC) method
for generalised projectile-target scattering, before describing its application
to the cases of: electron-impact hydrogen (e-H) scattering, and electron-impact
helium (e-He) scattering.
In particular, we shall explore the treatment of target ionisation within the CCC
method.
Note that in the general treatment, we shall restrict our attention to electron
projectiles and atomic/ionic targets.

\subsection{Convergent Close-Coupling Method}
\label{sec:ccc-method}

In brief, the CCC method utilises the method of basis expansion, discussed in
further detail in \autoref{app:basis-expansion}, to numerically solve the
Lippmann-Schwinger equation, for a projectile-target system, to yield the
transition amplitudes, which are convergent as the size of the basis is
increased.
The rate of convergence depends on many factors, such as the complexity and
geometry of the projectile-target system for example, as well as the choice of
basis used in the expansion.
Furthermore, by selecting a complete basis, ionisation transition amplitudes can
be treated in a similar manner to discrete excitation transition amplitudes.

\subsubsection{Laguerre Basis}
\label{sec:laguerre-basis}

To describe the target structure, the CCC method utilises a Laguerre basis,
$\lrset{\ket{\varphi_{i}}}_{i = 1}^{\infty}$, for the Hilbert space
$L^{2}\lr{\real^{3}}$, for which the coordinate-space representation is of the
form
\begin{equation}
  \label{eq:laguerre-basis}
  \bra{\boldsymbol{r}}
  \ket{\varphi_{i}}
  =
  \varphi_{i}\lr{r, \Omega}
  =
  \tfrac{1}{r}
  \xi_{k_{i}, l{i}}\lr{r}
  Y_{l_{i}}^{m_{i}}\lr{\Omega}
\end{equation}
where $Y_{l_{i}}^{m_{i}}\lr{\Omega}$ are the spherical harmonics, and where
$\xi_{k_{i}, l_{i}}\lr{r}$ are the Laguerre radial basis functions, which are of
the form
\begin{equation}
  \label{eq:laguerre-radial-basis}
  \xi_{k, l}\lr{r}
  =
  \sqrt
  {
    \dfrac
    {
      \lambda_{l}
      \lr{k - 1}!
    }
    {
      \lr{2l + 1 + k}!
    }
  }
  \lr{\lambda_{l} r}^{l + 1}
  \exponential\lr[\big]{- \tfrac{1}{2} \lambda_{l} r}
  L_{k - 1}^{2l + 2}\lr{\lambda_{l} r}
\end{equation}
where $\alpha_{l}$ is the exponential fall-off, for each $l$, and where
$L_{k - 1}^{2l + 2}\lr{\lambda_{l} r}$ are the associated Laguerre polynomials.
Note that we must have that
$k_{i} \in \lrset{1, 2, \dotsc}$,
$l_{i} \in \lrset{0, 1, \dotsc}$ and
$m_{i} \in \lrset{-\ell_{i}, \dotsc, \ell_{i}}$, for each
$i \in \lrset{1, 2, \dotsc}$.

This Laguerre basis is utilised due to: the Laguerre basis functions,
$\lrset{\varphi_{i}\lr{r, \Omega}}_{i = 1}^{\infty}$,forming a complete basis
for the Hilbert space $L^{2}\lr{\real^{3}}$ - shown in
\autoref{app:laguerre-completeness}, the short-range and long-range behaviour of
the radial basis functions being well suited to describing both target states
and providing a basis for expanding continuum states in, and because it allows
the matrix elements of certain operators to be calculated analytically.

Practically, we cannot utilise a a basis of infinite size.
Hence, we truncate the Laguerre radial basis,
$\lrset{\xi_{k, l}\lr{r}}_{k = 1}^{N_{l}}$, to a certain number of radial basis
functions, $N_{l}$, for each $l$, and we also truncate
$l \in \lrset{0, \dotsc, l_{max}}$,
limiting the maximum angular momentum we consider in our basis.
Hence, for a given value of $m$, we have a basis size of
\begin{equation}
  \label{eq:basis-size}
  N
  =
  \sum_{l = 0}^{l_{max}}
  N_{l}
  .
\end{equation}
In the limit as $N \to \infty$, the truncated basis will tend towards
completeness, and it is in this limit that we discuss the convergence of the
Convergent Close-Coupling method.

Further properties of the Laguerre basis are discussed in
\autoref{app:laguerre-basis}.

\subsubsection{Projectile-Target System}
\label{sec:projectile-target}

Possessing now a suitable basis to work with, we proceed to represent the
projectile-target system in this basis by the method of basis expansion.
We first construct the projectile states, and the target states, before
considering how to combine the two into a set of total states.

Recall that we restrict our attention to the case of an electron projectile, and
an atomic/ionic target consisting of $n_{\mathrm{e}}$ electrons.
We shall adopt the convention that the projectile electron space is denoted by
$\mathcal{H}_{0}$, the $m$-th target electron space by $\mathcal{H}_{m}$, for
$m = 1, \dotsc, n_{\mathrm{e}}$, and a one-electron space by
$\mathcal{H}_{\mathrm{e}}$.
Furthermore, operators which act on the $m$-th electron space (including the
projectile electron), will be indexed by $m$, for
$m = 0, 1, \dotsc, n_{\mathrm{e}}$.

\paragraph{Projectile States}
\label{sec:projectile-states}

The projectile states, $\ket{\boldsymbol{k}} \in \mathcal{H}_{P}$, where
$\mathcal{H}_{P} = \mathcal{H}_{0} = \mathcal{H}_{\mathrm{e}}$ is the projectile
electron space, are defined to be eigenstates of the free Hamiltonian; that is,
\begin{equation}
  \label{eq:projectile-hamiltonian}
  \hat{K}_{0}
  \ket{\boldsymbol{k}}
  =
  \tfrac{k^{2}}{2}
  \ket{\boldsymbol{k}}
  .
\end{equation}
It follows that the coordinate-space representation of projectile states are
plane waves; that is,
\begin{equation}
  \label{eq:projectile-states}
  \bra{\boldsymbol{r}}
  \ket{\boldsymbol{k}}
  =
  \lr{2 \pi}^{-\tfrac{3}{2}}
  \exponential\lr
  {
    \imath
    \boldsymbol{k}
    \vdot
    \boldsymbol{r}
  }
  .
\end{equation}

Further properties of the plane waves are discussed in
\autoref{app:plane-waves}.

\paragraph{Target States}
\label{sec:target-states}

\subparagraph{Target Space}
\label{sec:target-space}

For a one-electron atomic/ionic target (such as H, $\mathrm{He}^{+}, \dotsc$),
the space of target states, $\mathcal{H}_{T}$, is simply constructed from the
one-electron space of the target electron,
$\mathcal{H}_{1} = \mathcal{H}_{\mathrm{e}}$.
However, for a many-electron atomic/ionic target (such as He), the construction
of the space of target states requires a more nuanced approach.

Firstly, we note that as electrons are indistinguishable, the one-electron space
of each target electron must be identical; that is,
$\mathcal{H}_{m} = \mathcal{H}_{\mathrm{e}}$ for
$m = 1, \dotsc, n_{\mathrm{e}}$ where $n_{\mathrm{e}}$ is the number of target
electrons.
Furthermore, the space of target states, $\mathcal{H}_{T}$, must be constructed
from the spaces of target electron in a way which preserves the
indistinguishableness of each electron, and which adheres to Pauli's exclusion
principle - that no more than one electron can occupy a given state (including
spin).
This is achieved by constructing the space of target states from the
anti-symmetrised tensor product of the spaces of the target electrons,
\begin{equation}
  \label{eq:target-space}
  \mathcal{H}_{T}
  =
  \lrset[\Bigg]
  {
    \hat{A}_{n_{\mathrm{e}}}
    \ket{\psi}
    :
    \ket{\psi}
    \in
    \bigotimes_{m = 1}^{n_{\mathrm{e}}}
    \mathcal{H}_{\mathrm{e}}
  }
\end{equation}
where the operator, $\hat{A}_{n_{\mathrm{e}}}$, anti-symmetrises the tensor
product of $n_{\mathrm{e}}$ indistinguishable electron states.

\subparagraph{Target Hamiltonian}
\label{sec:target-hamiltonian}

The target Hamiltonian, for an atomic/ionic target with $n_{\mathrm{e}}$
electrons, is of the form
\begin{equation}
  \label{eq:target-hamiltonian}
  \hat{H}_{T}
  =
  \sum_{m = 1}^{n_{\mathrm{e}}}
  \hat{K}_{m}
  +
  \sum_{m = 1}^{n_{\mathrm{e}}}
  \hat{V}_{m}
  +
  \sum_{m = 1}^{n_{\mathrm{e}}}
  \sum_{n = m + 1}^{n_{\mathrm{e}}}
  \hat{V}_{m, n}
\end{equation}
where $\hat{K}_{m}$ and $\hat{V}_{m}$ are the target electron kinetic and
electron-nuclei potential operators, for $m = 1, \dotsc, n_{\mathrm{e}}$, and
where $\hat{V}_{m, n}$ are the electron-electron potential operators, for
$m, n = 1, \dotsc, n_{\mathrm{e}}$.

\subparagraph{Target Diagonalisation}
\label{sec:target-diagonalisation}

The target Hamiltonian, restricted to just one target electron,
\begin{equation}
  \label{eq:target-1e-hamiltonian}
  \hat{H}_{T, \mathrm{e}}^{}
  =
  \hat{K}_{1}
  +
  \hat{V}_{1}
\end{equation}
is expanded in a Laguerre basis, $\lrset{\ket{\varphi_{i}}}_{i = 1}^{N}$, and
diagonalised to yield a set of one-electron atomic orbitals
$\lrset{\ket*{\phi_{i}^{\lr{N}}}}_{i = 1}^{N}$, which are orthonormal and
satisfy
\begin{equation}
  \label{eq:target-1e-orbitals}
  \bra*{\phi_{i}^{\lr{N}}}
  \hat{H}_{T, \mathrm{e}}
  \ket*{\phi_{j}^{\lr{N}}}
  =
  \varepsilon_{i}^{\lr{N}}
  \delta_{i, j}
  .
\end{equation}
From these one-electron atomic orbitals, we generate a set of one-electron spin
orbitals, $\lrset{\ket*{\chi_{i}^{\lr{N}}}}_{i = 1}^{2N}$, for which
$\ket*{\chi_{2i - 1}^{\lr{N}}}$ and $\ket*{\chi_{2i}^{\lr{N}}}$ both correspond
to $\ket*{\phi_{i}^{\lr{N}}}$ but have spin projection $\tfrac{1}{2}$ and
$-\tfrac{1}{2}$ respectively.
These one-electron spin orbitals are then combined to construct Slater
determinants; for any selection of $n_{\mathrm{e}}$ one-electron spin orbitals
$\ket*{\chi_{a_{1}}^{\lr{N}}}, \dotsc, \ket*{\chi_{a_{n_{\mathrm{e}}}}^{\lr{N}}}
\in \lrset{\ket*{\chi_{i}^{\lr{N}}}}_{i = 1}^{2N}$, the Slater determinant of
these spin orbitals is of the form
\begin{equation}
  \label{eq:target-slater-determinant}
  \ket*
  {
    \chi_{a_{1}}^{\lr{N}},
    \dotsc,
    \chi_{a_{n_{\mathrm{e}}}}^{\lr{N}}
  }
  =
  \dfrac{1}{\sqrt{n_{\mathrm{e}}!}}
  \sum_{\sigma \in S_{n_{\mathrm{e}}}}
  \mathrm{sgn}\lr{\sigma}
  \ket*{\chi_{a_{\sigma\lr{1}}}^{\lr{N}}}
  \otimes
  \dots
  \otimes
  \ket*{\chi_{a_{\sigma\lr{n_{\mathrm{e}}}}}^{\lr{N}}}
  \in
  \mathcal{H}_{T}
\end{equation}
where $S_{n_{\mathrm{e}}}$ is the symmetric group on $n_{\mathrm{e}}$ elements,
the sum is taken over all permutations, $\sigma \in S_{n_{\mathrm{e}}}$, and
where $\mathrm{sgn}\lr{\sigma}$ is the sign of the permutation $\sigma$.
We note that Slater determinants are anti-symmetric under pairwise exchange of
any two orbitals, and hence adhere to Pauli's exclusion principle.

The true target states, $\lrset{\ket{\Phi_{i}}}_{i = 1}^{\infty}$, are then
approximated by expanding the many-electron target Hamiltonian, $\hat{H}_{T}$,
in a basis of Slater determinants,
\begin{equation}
  \label{eq:target-slater-determinant-basis}
  \lrset[\big]
  {
    \ket*
    {
      \chi_{a_{1}}^{\lr{N}},
      \dotsc,
      \chi_{a_{n_{\mathrm{e}}}}^{\lr{N}}
    }
    :
    a_{1}, \dotsc, a_{n_{\mathrm{e}}}
    \in
    \lrset{1, \dotsc, 2N}
  }
\end{equation}
and diagonalising to yield a set of target pseudostates,
$\lrset{\ket*{\Phi_{i}^{\lr{N}}}}_{i = 1}^{N_{T}}$, which are orthonormal and
satisfy
\begin{equation}
  \label{eq:target-one-electron-states}
  \bra*{\Phi_{i}^{\lr{N}}}
  \hat{H}_{T}
  \ket*{\Phi_{j}^{\lr{N}}}
  =
  \epsilon_{i}^{\lr{N}}
  \delta_{i, j}
  .
\end{equation}
Note that the number of target pseudostates, $N_{T}$, depends on the number of
Slater determinants utilised in the expansion of $\hat{H}_{T}$.
The process of selecting which Slater determinants to use in the expansion is
not trivial, and will be discussed in further detail in
\autoref{sec:he-frozen-core}.

Note also that the $\lr{N}$ superscript has been introduced to indicate that
these are not true eigenstates of the target Hamiltonian, only of its
representation in the truncated Laguerrre basis, and that these pseudostates and
their pseudoenergies are dependent on the size of the Laguerre basis utilised.
The diagonalisation procedure is discussed in further detail in
\autoref{app:diagonalisation}.

\subparagraph{Completeness of Target Pseudostates}
\label{sec:target-states-completeness}

As a result of the completeness of the Laguerre basis, the set of target
pseudostates will be separable into a set of bounded pseudostates which will
form an approximation of the true target discrete spectrum, and a set of
unbounded pseudostates which will provide a discretisation of the true continuum
of unbounded states.
In general, bounded states have negative energy and unbounded states have
positive energy, however this is not necessarily the case - a note which will be
relevant in the treatment of the meta-stable positive-energy discrete states of
helium.
The treatment of these meta-stable states of helium will be discussed in further
detail in \autoref{sec:he-meta-stable}, while for now we shall simply
distinguish bounded and unbounded states by the sign of their energy.
We order the target pseudostates by increasing pseudoenergy,
$\epsilon_{1}^{\lr{N}} < \dotsc < \epsilon_{N_{T}}^{\lr{N}}$, which allows us to
express the separability of the spectrum in the form
\begin{equation}
  \label{eq:target-spectrum-separable}
  \lrset{\ket*{\Phi_{i}^{\lr{N}}}}_{i = 1}^{N_{T}}
  =
  \lrset{\ket*{\Phi_{i}^{\lr{N}}}}_{i = 1}^{N_{B}}
  \cup
  \lrset{\ket*{\Phi_{i}^{\lr{N}}}}_{i = N_{B} + 1}^{N_{T}}
\end{equation}
where $\epsilon_{i}^{\lr{N}} < 0$ for $i = 1, \dotsc, N_{B}$, and where
$\epsilon_{i}^{\lr{N}} \geq 0$ for $i = N_{B} + 1, \dotsc, N_{T}$.
Note that $N_{B}$ is the number of bounded pseudostates, and we write
$N_{U} = N_{T} - N_{B}$ to represent the number of unbounded pseudostates, both of
which are dependent on $N$ by consequence of the construction of the target
pseudostates.

We note that the identity operator for the space of target states,
$\mathcal{H}_{T}$, can be represented in the form
\begin{equation}
  \label{eq:target-identity}
  \hat{I}_{T}
  =
  \sum_{i = 1}^{\infty}
  \ket*{\Phi_{i}}
  \bra*{\Phi_{i}}
  +
  \int_{\boldsymbol{q} : q^{2} \geq 0}
  \dd{\boldsymbol{q}}
  \ket*{\boldsymbol{q}}
  \bra*{\boldsymbol{q}}
\end{equation}
where $\lrset{\ket*{\Phi_{i}}}_{i = 1}^{\infty}$ is the true target discrete
spectrum and where $\lrset{\ket*{\boldsymbol{q}} : q^{2} \geq 0}$ is the true
continuum spectrum, of singly ionised target states.
Furthermore, we note that the projection operator for the target pseudostates,
$\hat{I}_{T}^{\lr{N}}$, is of the form
\begin{equation}
  \label{eq:target-projection}
  \hat{I}_{T}^{\lr{N}}
  =
  \sum_{i = 1}^{N_{T}}
  \ket*{\Phi_{i}^{\lr{N}}}
  \bra*{\Phi_{i}^{\lr{N}}}
  =
  \sum_{i = 1}^{N_{B}}
  \ket*{\Phi_{i}^{\lr{N}}}
  \bra*{\Phi_{i}^{\lr{N}}}
  +
  \sum_{i = N_{B} + 1}^{N_{T}}
  \ket*{\Phi_{i}^{\lr{N}}}
  \bra*{\Phi_{i}^{\lr{N}}}
\end{equation}
and so in the limit as $N \to \infty$, the sum over the bounded pseudostates
will converge to the sum over the true target discrete states
\begin{equation}
  \label{eq:target-projection-discrete}
  \lim_{N \to \infty}
  \sum_{i = 1}^{N_{B}}
  \ket*{\Phi_{i}^{\lr{N}}}
  \bra*{\Phi_{i}^{\lr{N}}}
  =
  \sum_{i = 1}^{\infty}
  \ket*{\Phi_{i}}
  \bra*{\Phi_{i}}
\end{equation}
and the sum over the unbounded pseudostates will converge to a discretisation of
the integral over the true continuum spectrum
\begin{equation}
  \label{eq:target-projection-continuum}
  \lim_{N \to \infty}
  \sum_{i = N_{B} + 1}^{N_{T}}
  \ket*{\Phi_{i}^{\lr{N}}}
  \bra*{\Phi_{i}^{\lr{N}}}
  =
  \int_{\boldsymbol{q} : q^{2} \geq 0}
  \dd{\boldsymbol{q}}
  \ket*{\boldsymbol{q}}
  \bra*{\boldsymbol{q}}
\end{equation}
recalling that $N_{T}$, and $N_{B}$ are both dependent on $N$.
Whence, it follows that projection operator for the target pseudostates
converges to the identity operator, for $\mathcal{H}_{T}$,
in the limit as $N \to \infty$; that is,
\begin{equation}
  \label{eq:target-projection-convergence}
  \lim_{N \to \infty}
  \hat{I}_{T}^{\lr{N}}
  =
  \hat{I}_{T}
\end{equation}
which can be equivalently stated as
\begin{equation}
  \label{eq:target-state-completeness}
  \lim_{N \to \infty}
  \mathrm{span}\lrset
  {
    \ket*{\Phi_{i}^{\lr{N}}}
  }_{i = 1}^{N_{T}}
  =
  \mathcal{H}_{T}
  .
\end{equation}

\paragraph{Total State}
\label{sec:total-state}

The total Hamiltonian of the projectile-target system, $\hat{H}$, is of the form
\begin{equation}
  \label{eq:total-hamiltonian}
  \hat{H}
  =
  \hat{H}_{T}
  +
  \hat{K}_{0}
  +
  \hat{V}_{0}
  +
  \sum_{m = 1}^{n_{\mathrm{e}}}
  \hat{V}_{0, m}
\end{equation}
where $\hat{H}_{T}$ is the target Hamiltonian, defined in
\autoref{eq:target-hamiltonian}, $\hat{K}_{0}$ is the projectile electron
kinetic operator, $\hat{V}_{0}$ is the projectile electron-nuclei potential
operator, and $\hat{V}_{0, m}$ are the projectile electron-target electron
potential operators.
The total state of the projectile-target system, $\ket*{\Psi^{\lr{+}}}$,
specified to have outgoing spherical-wave boundary conditions, is an eigenstate
of the total Hamiltonian, $\hat{H}$,
\begin{equation}
  \label{eq:total-state}
  \hat{H}
  \ket*{\Psi^{\lr{+}}}
  =
  E
  \ket*{\Psi^{\lr{+}}}
\end{equation}
with total energy $E$.
Since the construction of the total state will depend upon the target
pseudostates, $\lrset{\ket*{\psi_{i}^{\lr{N}}}}_{i = 1}^{N}$, obtained by the
diagonalisation of the target hamiltonian, \autoref{eq:target-states}, we shall
make explicit it's dependence on the size of the basis by writing,
$\ket*{\Psi^{\lr{N, +}}}$, and note that
\begin{equation}
  \label{eq:total-state-expansion}
  \ket*{\Psi^{\lr{+}}}
  =
  \lim_{N \to \infty}
  \ket*{\Psi^{\lr{N, +}}}
  .
\end{equation}
To ensure that the total state is anti-symmetric (that is, that it respects the
indistinguishableness of individual electrons, as well as Pauli's exclusion
principle), we construct it using a multichannel expansion of the form
\begin{equation}
  \label{eq:total-state-multichannel}
  \ket*{\Psi^{\lr{N, +}}}
  =
  \lrsq[\bigg]
  {
    1
    -
    \sum_{m = 1}^{n_{\mathrm{e}}}
    \hat{P}_{0, m}
  }
  \ket*{\psi^{\lr{N, +}}}
\end{equation}
where $\hat{P}_{0, m}$ are the electron exchange operators, for
$m = 1, \dotsc, n_{\mathrm{e}}$, exchanging the projectile electron state and
$m$-th target electron state.
We neglect to anti-symmetrise the target electrons with each other, since the
target pseudostates, $\lrset{\ket*{\phi_{i}^{\lr{N}}}}_{i = 1}^{\infty}$ are
already anti-symmetric by construction.
However, we observe that the multichannel expansion is not uniquely defined,
since for any state in the kernel of the multichannel operator,
$\ket*{\omega^{\lr{N, +}}} \in
\ker\lr{1 - \sum_{m = 1}^{n_{\mathrm{e}}}\hat{P}_{0, m}}$,
and constant $\theta$, the multichannel expansion of
$\ket*{\psi^{\lr{N, +}}} + \theta\ket*{\omega^{\lr{N, +}}}$ will be identical to
that of $\ket*{\psi^{\lr{N, +}}}$.
This dilemma can be resolved in the following
To resolve this dilemma, we impose the constraint that for any of the
one-electron states, $\ket*{\eta}$, used to construct the target state,
$\ket*{\phi_{i}^{\lr{N}}}$, that
\begin{equation}
  \label{eq:multichannel-constraint}
  \bra*{\eta, \phi_{i}^{\lr{N}}}
  \hat{P}_{0, m}
  \ket*{\gamma^{\lr{N, +}}}
  =
  -
  \bra*{\eta, \phi_{i}^{\lr{N}}}
  \ket*{\gamma^{\lr{N, +}}}
  \qq{for all}
  \ket*{\gamma^{\lr{N, +}}}
  .
\end{equation}
Whence it follows that the multichannel expansion of
$\ket*{\psi^{\lr{N, +}}} + \theta\ket*{\omega^{\lr{N, +}}}$ will be identical to
that of $\ket*{\psi^{\lr{N, +}}}$, only in the case where $\theta = 0$, or
$\ket*{\omega} = \ket{0}$; that is to say that $\ket*{\psi^{\lr{N, +}}}$ will
now uniquely determine $\ket*{\Psi^{\lr{N, +}}}$.

\subsubsection{Close-Coupling Equations}
\label{sec:cc-equations}

\subsubsection{Transition Amplitudes}
\label{sec:transition-amplitudes}

\subsubsection{Cross Sections}
\label{sec:cross-sections}

\paragraph{Total Cross Sections}
\label{sec:cc-total-cross-sections}

\paragraph{Differential Cross Sections}
\label{sec:cc-differential-cross-sections}

\subsubsection{S-Wave Model}
\label{sec:s-wave-model}

\subsection{Electron-Impact Hydrogen Scattering}
\label{sec:e-h}

\subsubsection{Elastic Scattering}
\label{sec:e-h-elastic-scattering}

\subsubsection{Excitation}
\label{sec:e-h-excitation}

\subsubsection{Ionisation}
\label{sec:e-h-ionisation}

\paragraph{Singlet Case}
\label{sec:e-h-singlet}

\paragraph{Triplet Case}
\label{sec:e-h-triplet}

\subsection{Electron-Impact Helium Scattering}
\label{sec:e-he}

\subsubsection{Additional Considerations for a Helium Target}
\label{sec:he-target}

\paragraph{Frozen-Core Model}
\label{sec:he-frozen-core}

\paragraph{Meta-stable States}
\label{sec:he-meta-stable}

\todo[meta-stable states]{
  How are positive energy discrete states handled?
  Do positive energy discrete states overlap with the continuum?
  Do we simply include the positive energy discrete states in the continuum with
  a Dirac mass function?
}

\subsubsection{Elastic Scattering}
\label{sec:e-he-elastic-scattering}

\subsubsection{Excitation}
\label{sec:e-he-excitation}

\paragraph{Auto-Ionisation}
\label{sec:e-he-auto-ionisation}

\subsubsection{Ionisation}
\label{sec:e-he-ionisation}

\section{Survey of Experimental Literature}
\label{sec:survey-experimental}

\section{Survey of Theoretical Literature}
\label{sec:survey-theoretical}

\subsection{Electron-Impact Hydrogen Ionisation Calculations}
\label{sec:e-h-ionisation-calculations}

\subsubsection{Convergent Close-Coupling Calculations}
\label{sec:e-h-ccc-calculations}

\subsubsection{Exterior-Complex-Scaling Calculations}
\label{sec:e-h-ecs-calculations}

\subsubsection{Ansatz of Zatsarinny and Bartschat}
\label{sec:e-h-ecs-calculations}

\subsection{Electron-Impact Helium Ionisation Calculations}
\label{sec:e-he-ionisation-calculations}

\subsubsection{Convergent Close-Coupling Calculations}
\label{sec:e-he-ccc-calculations}

\subsubsection{Exterior-Complex-Scaling Calculations}
\label{sec:e-he-ecs-calculations}

\subsubsection{Ansatz of Zatsarinny and Bartschat}
\label{sec:e-he-ecs-calculations}

\section{Conclusion}
\label{sec:conclusion}

\clearpage

\bibliographystyle{chicago}

\bibliography{references}

\clearpage

\appendix

\section{Properties of Utilised Bases}
\label{app:properties}

\subsection{Spherical Harmonics}
\label{app:spherical-harmonics}

\subsubsection{Completeness}
\label{app:spherical-harmonic-completeness}

\todo[spherical harmonic completeness]{
  Prove that the set of spherical harmonics forms,
  $\lrset{Y_{l}^{-l}\lr{\Omega}, \dotsc, Y_{l}^{l}\lr{\Omega}}_{l = 0}^{\infty}$,
  forms an orthonormal, complete basis for the Hilbert space
  $L^{2}\lr{S^{2}}$.
}

\subsection{Laguerre Radial Basis}
\label{app:laguerre-radial-basis}

\subsubsection{Completeness}
\label{app:laguerre-radial-completeness}

\todo[laguerre radial completeness]{
  Prove that the Laguerre radial basis functions,
  $\lrset{\xi_{k, l}\lr{r}}_{k = 1}^{\infty}$, for each $l$, forms a complete
  basis for the Hilbert space $L^{2}\lr{[0, \infty)}$.
}

\subsection{Laguerre Basis}
\label{app:laguerre-basis}

\subsubsection{Completeness}
\label{app:laguerre-completeness}

It is shown in \autoref{app:laguerre-radial-completeness}, that the Laguerre
radial basis functions, $\lrset{\xi_{k, l}\lr{r}}_{k = 1}^{\infty}$, for each
$l$, forms a complete basis for the Hilbert space $L^{2}\lr{[0, \infty)}$.
Similarly, it is also shown in \autoref{app:spherical-harmonic-completeness},
that the set of spherical harmonics,
$\lrset{Y_{l}^{-l}\lr{\Omega}, \dotsc, Y_{l}^{l}\lr{\Omega}}_{l = 0}^{\infty}$,
forms an orthonormal, complete basis for the Hilbert space $L^{2}\lr{S^{2}}$.
Hence, the Laguerre basis functions
$\lrset{\varphi_{i}\lr{r, \Omega}}_{i = 1}^{\infty}$, form a complete basis
for the Hilbert space $L^{2}\lr{\real^{3}}$.

\subsection{Plane Waves}
\label{app:plane-waves}

\section{Numerical Techniques}
\label{app:numerical-techniques}

\subsection{Basis Expansion}
\label{app:basis-expansion}

\todo[basis expansion]{
  Elaborate on the method of basis expansion for Hilbert spaces.
}

\subsection{Diagonalisation}
\label{app:diagonalisation}

\todo[target diagonalisation]{
  Elaborate on the diagonalisation procedure for the target states.
}

\clearpage

\todos

\end{document}