\documentclass[draft]{article}

% - Style
\usepackage{base}

% - Title
\gdef\theassessment{PHYS4001 - Literature Review}
\gdef\thesupervisor{Professor Igor Bray}
\gdef\theinstitution{Curtin University}
\gdef\thestudentid{1834 2884}

\title{Ionisation Amplitudes in Electron-Impact Helium Collisions within the
  S-Wave Model}
\author{Tom Ross}
\date{\today}

% - Headers
\pagestyle{fancy}
\fancyhf{}
\rhead{\theauthor}
\chead{}
\lhead{\theassessment}
\rfoot{\thepage}
\cfoot{}
\lfoot{}

% document

\begin{document}

% cover page

\begin{titlepage}
  \begin{flushleft}
    \theinstitution \hfill \theassessment
  \end{flushleft}
  \hrule
  \begin{center}
    {
      \huge
      \thetitle
    }
    ~\\
    \rule[1.0pt]{8.5cm}{0.4pt}
    ~\\
    {
      \large
      \theauthor ~supervised by \thesupervisor
    }
  \end{center}
  \hrule
  \begin{center}
    [ABSTRACT]
  \end{center}
\end{titlepage}

\clearpage

% contents page

\tableofcontents

\listoffigures

\listoftables

% statement

\clearpage

\section{Introduction}
\label{sec:introduction}

\paragraph{Applications of Electron-Impact Hydrogen Scattering}

\paragraph{Specific Applications of Electron-Impact Hydrogen Ionisation}

\paragraph{Development of Quantum Scattering Theory}

\section{Theory}
\label{sec:theory}

We shall describe the development of the Convergent Close-Coupling (CCC) method
for generalised projectile-target scattering, before describing its application
to the cases of: electron-impact hydrogen (e-H) scattering, and electron-impact
helium (e-He) scattering.
In particular, we shall explore the treatment of target ionisation within the CCC
method.

\subsection{Convergent Close-Coupling Method}
\label{sec:ccc-method}

In brief, the CCC method utilises the method of basis expansion to numerically
solve the Lippmann-Schwinger equation, for a projectile-target system, to yield
the transition amplitudes, which are convergent as the size of the basis is
increased.
The rate of convergence depends on many factors, such as the complexity and
geometry of the projectile-target system for example, as well as the choice of
basis used in the expansion.
Furthermore, by selecting a complete basis, ionisation transition amplitudes can
be treated in a similar manner to discrete excitation transition amplitudes.

\subsubsection{Laguerre Basis}
\label{sec:laguerre-basis}

To describe the target structure, the CCC method utilises a Laguerre basis,
$\lrset{\ket{\varphi_{i}}}_{i = 1}^{\infty}$, for which the coordinate-space
representation is of the form
\begin{equation}
  \label{eq:laguerre-basis}
  \bra{\boldsymbol{r}}
  \ket{\varphi_{i}}
  =
  \tfrac{1}{r}
  \xi_{k_{i}, l{i}}\lr{r}
  Y_{l_{i}}^{m_{i}}\lr{\Omega}
\end{equation}
where $Y_{l_{i}}^{m_{i}}\lr{\Omega}$ are the spherical harmonics, and where
$\xi_{k_{i}, l_{i}}\lr{r}$ are the Laguerre radial basis functions, which are of
the form
\begin{equation}
  \label{eq:laguerre-radial-basis}
  \xi_{k, l}\lr{r}
  =
  \sqrt
  {
    \dfrac
    {
      \lambda_{l}
      \lr{k - 1}!
    }
    {
      \lr{2l + 1 + k}!
    }
  }
  \lr{\lambda_{l} r}^{l + 1}
  \exponential\lr[\big]{- \tfrac{1}{2} \lambda_{l} r}
  L_{k - 1}^{2l + 2}\lr{\lambda_{l} r}
\end{equation}
where $\alpha_{l}$ is the exponential fall-off, for each $l$, and where
$L_{k - 1}^{2l + 2}\lr{\lambda_{l} r}$ are the associated Laguerre polynomials.
Note that we must have that
$k_{i} \in \lrset{1, 2, \dotsc}$,
$l_{i} \in \lrset{0, 1, \dotsc}$ and
$m_{i} \in \lrset{-\ell_{i}, \dotsc, \ell_{i}}$, for each
$i \in \lrset{1, 2, \dotsc}$.

It is shown in \autoref{app:laguerre-completeness}, for each $l$, that the
Laguerre radial basis functions, $\lrset{\xi_{k, l}\lr{r}}_{k = 1}^{\infty}$,
forms a complete basis for the Hilbert space $L_{\complex}^{2}\lr{[0, \infty)}$.
Similarly, it is also shown in \autoref{app:spherical-harmonic-completeness},
that the set of spherical harmonics,
$\lrset{Y_{l}^{-l}\lr{\Omega}, \dotsc, Y_{l}^{l}\lr{\Omega}}_{l = 0}^{\infty}$,
forms a orthonormal, complete basis for the Hilbert space
$L_{\complex}^{2}\lr{S^{2}}$.
Hence, the Laguerre basis functions
$\lrset{\varphi_{i}\lr{r, \Omega}}_{i = 1}^{\infty}$, forms a complete basis
for the Hilbert space $L_{\complex}^{2}\lr{\real^{3}}$ space.

Practically, we cannot utilise a a basis of infinite size.
Hence, we truncate the Laguerre radial basis,
$\lrset{\xi_{k, l}\lr{r}}_{k = 1}^{N_{l}}$, to a certain number of radial basis
functions, $N_{l}$, for each $l$, and we also truncate
$l \in \lrset{0, \dotsc, l_{max}}$,
limiting the maximum angular momentum we consider in our basis.
Hence, for a given value of $m$, we have a basis size of
\begin{equation}
  \label{eq:basis-size}
  N
  =
  \sum_{l = 0}^{l_{max}}
  N_{l}
  .
\end{equation}
In the limit as $N \to \infty$, the truncated basis will tend towards
completeness, and it is in this limit that we discuss the convergence of the
Convergent Close-Coupling method.

Further properties of the Laguerre basis are provided in
\autoref{app:laguerre-basis}.

\subsubsection{Projectile-Target System}
\label{sec:projectile-target}

Possessing now a suitable basis to work with, we proceed to represent the
projectile-target system in this basis by the method of basis expansion.
As a result of the asymptotic initial conditions, in which the projectile and
the target are assumed to be sufficiently far apart as to be non-interacting,
the total state of the projectile-target system can be constructed as a tensor
product of the individual projectile and target states.
That is, the total state, $\ket{\Phi}$, may be written as
$\ket{\Phi} = \ket{\phi, \boldsymbol{k}}$,
where $\ket{\phi}$ is the target state, and $\ket{\boldsymbol{k}}$ is the
projectile state.
It now remains to construct the projectile states, and the target states.

\paragraph{Projectile States}

The projectile states, $\ket{\boldsymbol{k}}$, are defined to be eigenstates of
the free Hamiltonian; that is,
\begin{equation}
  \label{eq:projectile-hamiltonian}
  \hat{H}_{1}
  \ket{\boldsymbol{k}}
  =
  \hat{K}_{1}
  \ket{\boldsymbol{k}}
  =
  \tfrac{k^{2}}{2}
  \ket{\boldsymbol{k}}
  .
\end{equation}
It follows that the projectile states are plane waves, for which the
coordinate-space representation is of the form
\begin{equation}
  \label{eq:projectile-states}
  \bra{\boldsymbol{r}}
  \ket{\boldsymbol{k}}
  =
  \lr{2 \pi}^{-\tfrac{3}{2}}
  \exponential\lr
  {
    \imath
    \boldsymbol{k}
    \vdot
    \boldsymbol{r}
  }
  .
\end{equation}

\paragraph{Target Structure}

The target states, $\ket{\phi}$ are constructed by expanding the target
Hamiltonian,
\begin{equation}
  \label{eq:target-hamiltonian}
  \hat{H}_{2}
  \ket{\phi}
  =
  \lrsq
  {
    \hat{K}_{2}
    +
    \hat{V}_{2}
  }
  \ket{\phi}
\end{equation}
in a Laguerre basis, and diagonalising to yield the target pseudostates
$\lrset{\ket{\phi_{i}}}_{i = 1}^{N}$

\subsubsection{Close-Coupling Equations}
\label{sec:cc-equations}

\subsubsection{Transition Amplitudes}
\label{sec:transition-amplitudes}

\subsubsection{Cross Sections}
\label{sec:cross-sections}

\paragraph{Total Cross Sections}

\paragraph{Differential Cross Sections}

\subsubsection{S-Wave Model}
\label{sec:s-wave-model}

\subsection{Electron-Impact Hydrogen Scattering}
\label{sec:e-h}

\subsubsection{Elastic Scattering}
\label{sec:e-h-elastic-scattering}

\subsubsection{Excitation}
\label{sec:e-h-excitation}

\subsubsection{Ionisation}
\label{sec:e-h-ionisation}

\paragraph{Singlet Case}

\paragraph{Triplet Case}

\subsection{Electron-Impact Helium Scattering}
\label{sec:e-he}

\subsubsection{Considerations for a Two-Electron Target}
\label{sec:2e-target}

\paragraph{Pauli Exclusion Principle}

\paragraph{Frozen-Core Model}

\subsubsection{Elastic Scattering}
\label{sec:e-he-elastic-scattering}

\subsubsection{Excitation}
\label{sec:e-he-excitation}

\paragraph{Auto-Ionisation}

\subsubsection{Ionisation}
\label{sec:e-he-ionisation}

\section{Survey of Experimental Literature}
\label{sec:survey-experimental}

\section{Survey of Theoretical Literature}
\label{sec:survey-theoretical}

\subsection{Electron-Impact Hydrogen Ionisation Calculations}
\label{sec:e-h-ionisation-calculations}

\subsubsection{Convergent Close-Coupling Calculations}
\label{sec:e-h-ccc-calculations}

\subsubsection{Exterior-Complex-Scaling Calculations}
\label{sec:e-h-ecs-calculations}

\subsubsection{Ansatz of Zatsarinny and Bartschat}
\label{sec:e-h-ecs-calculations}

\subsection{Electron-Impact Helium Ionisation Calculations}
\label{sec:e-he-ionisation-calculations}

\subsubsection{Convergent Close-Coupling Calculations}
\label{sec:e-he-ccc-calculations}

\subsubsection{Exterior-Complex-Scaling Calculations}
\label{sec:e-he-ecs-calculations}

\subsubsection{Ansatz of Zatsarinny and Bartschat}
\label{sec:e-he-ecs-calculations}

\section{Conclusion}
\label{sec:conclusion}

\clearpage

\bibliographystyle{chicago}
\section{Laguerre Radial Basis}
\label{sec:laguerre-radial-basis}

\bibliography{references}

\clearpage

\appendix

\section{Properties of Laguerre Basis}
\label{app:laguerre-basis}

\subsection{Laguerre Radial Basis}
\label{app:laguerre-radial-basis}

\subsubsection{Completeness}
\label{app:laguerre-completeness}

\subsection{Spherical Harmonic}
\label{app:spherical-harmonics}

\subsubsection{Completeness}
\label{app:spherical-harmonic-completeness}


\end{document}