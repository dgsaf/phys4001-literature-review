\documentclass[draft]{article}

% - Style
\usepackage{base}

% - Title
\gdef\theassessment{PHYS4001 - Literature Review}
\gdef\thesupervisor{Professor Igor Bray}
\gdef\theinstitution{Curtin University}
\gdef\thestudentid{1834 2884}

\title{Ionisation Amplitudes in Electron-Impact Helium Collisions within the
  S-Wave Model}
\author{Tom Ross}
\date{\today}

% - Headers
\pagestyle{fancy}
\fancyhf{}
\rhead{\theauthor}
\chead{}
\lhead{\theassessment}
\rfoot{\thepage}
\cfoot{}
\lfoot{}

% document

\begin{document}

% cover page

\begin{titlepage}
  {
    \centering
    {
      \scshape\large\theinstitution
      ~\\
      \scshape\large\theassessment
      \par
    }
    \vspace{1 em}
    \hrule
    ~\\
    {
      \scshape\huge\thetitle
      \par
    }
    \vspace{1 em}
    {
      \scshape\large
      {
        \theauthor
        ~supervised by
        \thesupervisor
      }
      \par
    }
    ~\\
    \hrule
  }
  \vspace{2 em}
  {
    [ABSTRACT]
  }
\end{titlepage}

\clearpage

% contents page

\tableofcontents

\listoffigures

\listoftables

% statement

\clearpage

\section{Introduction}
\label{sec:introduction}

\paragraph{Applications of Electron-Impact Hydrogen Scattering}

\paragraph{Specific Applications of Electron-Impact Hydrogen Ionisation}

\paragraph{Development of Quantum Scattering Theory}

\section{Theory}
\label{sec:theory}

\subsection{Convergent Close-Coupling Method}
\label{sec:ccc-method}

\subsubsection{Laguerre Basis}
\label{sec:laguerre-basis}

\subsubsection{Target-Projectile System}
\label{sec:target-projectile}

\subsubsection{Close-Coupling Equations}
\label{sec:cc-equations}

\subsubsection{Transition Amplitudes}
\label{sec:transition-amplitudes}

\subsubsection{Cross Sections}
\label{sec:cross-sections}

\paragraph{Total Cross Sections}

\paragraph{Differential Cross Sections}

\subsection{Electron-Impact Hydrogen Scattering}
\label{sec:e-h}

\subsubsection{Elastic Scattering}
\label{sec:e-h-elastic-scattering}

\subsubsection{Excitation}
\label{sec:e-h-excitation}

\subsubsection{Ionisation}
\label{sec:e-h-ionisation}

\paragraph{Singlet Case}

\paragraph{Triplet Case}

\subsection{Electron-Impact Helium Scattering}
\label{sec:e-he}

\subsubsection{Considerations for a Two-Electron Target}
\label{sec:2e-target}

\paragraph{Pauli Exclusion Principle}

\paragraph{Frozen-Core Model}

\subsubsection{Elastic Scattering}
\label{sec:e-he-elastic-scattering}

\subsubsection{Excitation}
\label{sec:e-he-excitation}

\paragraph{Auto-Ionisation}

\subsubsection{Ionisation}
\label{sec:e-he-ionisation}

\section{Survey of Experimental Literature}
\label{sec:survey-experimental}

\section{Survey of Theoretical Literature}
\label{sec:survey-theoretical}

\subsection{Electron-Impact Hydrogen Ionisation Calculations}
\label{sec:e-h-ionisation-calculations}

\subsubsection{Convergent Close-Coupling Calculations}
\label{sec:e-h-ccc-calculations}

\subsubsection{Exterior-Complex-Scaling Calculations}
\label{sec:e-h-ecs-calculations}

\subsubsection{Ansatz of Zatsarinny and Bartschat}
\label{sec:e-h-ecs-calculations}

\subsection{Electron-Impact Helium Ionisation Calculations}
\label{sec:e-he-ionisation-calculations}

\subsubsection{Convergent Close-Coupling Calculations}
\label{sec:e-he-ccc-calculations}

\subsubsection{Exterior-Complex-Scaling Calculations}
\label{sec:e-he-ecs-calculations}

\subsubsection{Ansatz of Zatsarinny and Bartschat}
\label{sec:e-he-ecs-calculations}

\section{Conclusion}
\label{sec:conclusion}

\clearpage

\bibliographystyle{chicago}
\bibliography{references}

\end{document}