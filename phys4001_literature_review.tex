\documentclass[draft]{article}

% - Style
\usepackage{base}

% - Title
\gdef\theassessment{PHYS4001 - Literature Review}
\gdef\thesupervisor{Professor Igor Bray}
\gdef\theinstitution{Curtin University}
\gdef\thestudentid{1834 2884}

\title{Ionisation Amplitudes in Electron-Impact Helium Collisions within the
  S-Wave Model}
\author{Tom Ross}
\date{\today}

% - Headers
\pagestyle{fancy}
\fancyhf{}
\rhead{\theauthor}
\chead{}
\lhead{\theassessment}
\rfoot{\thepage}
\cfoot{}
\lfoot{}

% document

\begin{document}

% cover page

\begin{titlepage}
  \begin{flushleft}
    \theinstitution \hfill \theassessment
  \end{flushleft}
  \hrule
  \begin{center}
    {
      \huge
      \thetitle
    }
    ~\\
    \rule[1.0pt]{8.5cm}{0.4pt}
    ~\\
    {
      \large
      \theauthor ~supervised by \thesupervisor
    }
  \end{center}
  \hrule
  \begin{center}
    [ABSTRACT]
  \end{center}
\end{titlepage}

\clearpage

% contents page

\tableofcontents

\listoffigures

\listoftables

% statement

\clearpage

\section{Introduction}
\label{sec:introduction}

\paragraph{Applications of Electron-Impact Hydrogen Scattering}

\paragraph{Specific Applications of Electron-Impact Hydrogen Ionisation}

\paragraph{Development of Quantum Scattering Theory}

\section{Theory}
\label{sec:theory}

We shall describe the development of the Convergent Close-Coupling (CCC) method
for generalised projectile-target scattering, before describing its application
to the cases of: electron-impact hydrogen (e-H) scattering, and electron-impact
helium (e-He) scattering.
In particular, we shall explore the treatment of target ionisation within the CCC
method.

\subsection{Convergent Close-Coupling Method}
\label{sec:ccc-method}

In brief, the CCC method utilises the method of basis expansion, discussed in
further detail in \autoref{app:basis-expansion}, to numerically solve the
Lippmann-Schwinger equation, for a projectile-target system, to yield the
transition amplitudes, which are convergent as the size of the basis is
increased.
The rate of convergence depends on many factors, such as the complexity and
geometry of the projectile-target system for example, as well as the choice of
basis used in the expansion.
Furthermore, by selecting a complete basis, ionisation transition amplitudes can
be treated in a similar manner to discrete excitation transition amplitudes.

\subsubsection{Laguerre Basis}
\label{sec:laguerre-basis}

To describe the target structure, the CCC method utilises a Laguerre basis,
$\lrset{\ket{\varphi_{i}}}_{i = 1}^{\infty}$, for which the coordinate-space
representation is of the form
\begin{equation}
  \label{eq:laguerre-basis}
  \bra{\boldsymbol{r}}
  \ket{\varphi_{i}}
  =
  \tfrac{1}{r}
  \xi_{k_{i}, l{i}}\lr{r}
  Y_{l_{i}}^{m_{i}}\lr{\Omega}
\end{equation}
where $Y_{l_{i}}^{m_{i}}\lr{\Omega}$ are the spherical harmonics, and where
$\xi_{k_{i}, l_{i}}\lr{r}$ are the Laguerre radial basis functions, which are of
the form
\begin{equation}
  \label{eq:laguerre-radial-basis}
  \xi_{k, l}\lr{r}
  =
  \sqrt
  {
    \dfrac
    {
      \lambda_{l}
      \lr{k - 1}!
    }
    {
      \lr{2l + 1 + k}!
    }
  }
  \lr{\lambda_{l} r}^{l + 1}
  \exponential\lr[\big]{- \tfrac{1}{2} \lambda_{l} r}
  L_{k - 1}^{2l + 2}\lr{\lambda_{l} r}
\end{equation}
where $\alpha_{l}$ is the exponential fall-off, for each $l$, and where
$L_{k - 1}^{2l + 2}\lr{\lambda_{l} r}$ are the associated Laguerre polynomials.
Note that we must have that
$k_{i} \in \lrset{1, 2, \dotsc}$,
$l_{i} \in \lrset{0, 1, \dotsc}$ and
$m_{i} \in \lrset{-\ell_{i}, \dotsc, \ell_{i}}$, for each
$i \in \lrset{1, 2, \dotsc}$.

It is shown in \autoref{app:laguerre-completeness}, for each $l$, that the
Laguerre radial basis functions, $\lrset{\xi_{k, l}\lr{r}}_{k = 1}^{\infty}$,
forms a complete basis for the Hilbert space $L_{\complex}^{2}\lr{[0, \infty)}$.
Similarly, it is also shown in \autoref{app:spherical-harmonic-completeness},
that the set of spherical harmonics,
$\lrset{Y_{l}^{-l}\lr{\Omega}, \dotsc, Y_{l}^{l}\lr{\Omega}}_{l = 0}^{\infty}$,
forms a orthonormal, complete basis for the Hilbert space
$L_{\complex}^{2}\lr{S^{2}}$.
Hence, the Laguerre basis functions
$\lrset{\varphi_{i}\lr{r, \Omega}}_{i = 1}^{\infty}$, forms a complete basis
for the Hilbert space $L_{\complex}^{2}\lr{\real^{3}}$ space.

This Laguerre basis is utilised due to being a complete basis, the short-range
and long-range behaviour of the radial basis functions, and because it allows
the matrix elements of certain operators to be calculated analytically.

Practically, we cannot utilise a a basis of infinite size.
Hence, we truncate the Laguerre radial basis,
$\lrset{\xi_{k, l}\lr{r}}_{k = 1}^{N_{l}}$, to a certain number of radial basis
functions, $N_{l}$, for each $l$, and we also truncate
$l \in \lrset{0, \dotsc, l_{max}}$,
limiting the maximum angular momentum we consider in our basis.
Hence, for a given value of $m$, we have a basis size of
\begin{equation}
  \label{eq:basis-size}
  N
  =
  \sum_{l = 0}^{l_{max}}
  N_{l}
  .
\end{equation}
In the limit as $N \to \infty$, the truncated basis will tend towards
completeness, and it is in this limit that we discuss the convergence of the
Convergent Close-Coupling method.

Further properties of the Laguerre basis are provided in
\autoref{app:laguerre-basis}.

\subsubsection{Projectile-Target System}
\label{sec:projectile-target}

Possessing now a suitable basis to work with, we proceed to represent the
projectile-target system in this basis by the method of basis expansion.
We first construct the projectile states, and the target states, before
considering how to combine the two into a set of total states.

\paragraph{Projectile States}

The projectile states, $\ket{\boldsymbol{k}}$, are defined to be eigenstates of
the free Hamiltonian; that is,
\begin{equation}
  \label{eq:projectile-free-hamiltonian}
  \hat{K}_{1}
  \ket{\boldsymbol{k}}
  =
  \tfrac{k^{2}}{2}
  \ket{\boldsymbol{k}}
  .
\end{equation}
It follows that the projectile states are plane waves, for which the
coordinate-space representation is of the form
\begin{equation}
  \label{eq:projectile-states}
  \bra{\boldsymbol{r}}
  \ket{\boldsymbol{k}}
  =
  \lr{2 \pi}^{-\tfrac{3}{2}}
  \exponential\lr
  {
    \imath
    \boldsymbol{k}
    \vdot
    \boldsymbol{r}
  }
  .
\end{equation}
Note that the Hamiltonian for the projectile, in the projectile-target system,
is of the form
\begin{equation}
  \label{eq:projectile-hamiltonian}
  \hat{H}_{1}
  \ket{\boldsymbol{k}}
  =
  \lrsq
  {
    \hat{K}_{1}
    +
    \hat{V}_{1}
  }
  \ket{\boldsymbol{k}}
  =
  \lrsq
  {
    \tfrac{k^{2}}{2}
    +
    \hat{V}_{1}
  }
  \ket{\boldsymbol{k}}
  .
\end{equation}

\paragraph{Target States}

The target states, $\ket{\phi}$ are constructed by expanding the target
Hamiltonian, $\hat{H}_{2} = \hat{K}_{2} + \hat{V}_{2}$,
in a Laguerre basis, $\lrset{\ket{\varphi_{i}}}_{i = 1}^{N}$, and diagonalising
to yield the target pseudostates
$\lrset{\ket*{\phi_{i}^{\lr{N}}}}_{i = 1}^{N}$, which are orthonormal and satisfy
\begin{equation}
  \label{eq:target-hamiltonian}
  \bra*{\phi_{i}^{\lr{N}}}
  \hat{H}_{2}
  \ket*{\phi_{j}^{\lr{N}}}
  =
  \epsilon_{i}^{\lr{N}}
  \delta_{i, j}
  .
\end{equation}
Note that the $\lr{N}$ superscript has been introduced to indicate that these
are not true eigenstates of the target Hamiltonian, only of its representation
in the truncated Laguerrre basis, and that these pseudostates and their
pseudoenergies are dependent on the size of the Laguerre basis utilised.
The procedure of diagonalising the target Hamiltonian is discussed in further
detail in \autoref{app:diagonalisation}.

As a result of the completeness of the Laguerre basis, the set of target
pseudostates will be separable into a set of bounded pseudostates which will
form an approximation of the true target discrete spectrum, and a set of
unbounded pseudostates which will provide a discretisation of the true continuum
of unbounded states.
In general, bounded states have negative energy and unbounded states have
positive energy, however this is not necessarily the case - a note which will be
relevant in the treatment of the meta-stable positive-energy discrete states of
helium.
\todo{How to handle positive energy discrete states?}
\todo{Do positive energy discrete states overlap with the continuum?}
For clarity, we shall adjourn the treatment of these meta-stable states until
required, and proceed with the assumption that bounded states have negative
energy.
We order the target pseudostates by increasing pseudoenergy,
$\epsilon_{1}^{\lr{N}} < \dotsc < \epsilon_{N}^{\lr{N}}$, which allows us to
express the separability of the spectrum in the form
\begin{equation}
  \label{eq:target-spectrum-separable}
  \lrset{\ket*{\phi_{i}^{\lr{N}}}}_{i = 1}^{N}
  =
  \lrset{\ket*{\phi_{i}^{\lr{N}}}}_{i = 1}^{N_{B}}
  \cup
  \lrset{\ket*{\phi_{i}^{\lr{N}}}}_{i = N_{B} + 1}^{N}
\end{equation}
where $\epsilon_{i}^{\lr{N}} < 0$ for $i = 1, \dotsc, N_{B}$, and where
$\epsilon_{i}^{\lr{N}} \geq 0$ for $i = N_{B} + 1, \dotsc, N$.
Note that $N_{B}$ is the number of bounded pseudostates, and we write
$N_{U} = N - N_{B}$ to represent the number of unbounded pseudostates, both of
which are dependent on $N$ by consequence of the diagonalisation procedure.

We note that the identity operator for the space of target states can be
represented in the form
\begin{equation}
  \label{eq:target-identity}
  \hat{I}_{2}
  =
  \sum_{i = 1}^{\infty}
  \ket*{\phi_{i}}
  \bra*{\phi_{i}}
  +
  \int_{\boldsymbol{q} : q^{2} \geq 0}
  \dd{\boldsymbol{q}}
  \ket*{\boldsymbol{q}}
  \bra*{\boldsymbol{q}}
\end{equation}
where $\lrset{\ket*{\phi_{i}}}_{i = 1}^{\infty}$ is the true target discrete
spectrum and where $\lrset{\ket*{\boldsymbol{q}} : q^{2} \geq 0}$ is the true
continuum spectrum.
\todo{Do we simply include the positive energy discrete states in the continuum
  with a Dirac mass function?}
Furthermore, we note that the projection operator for the target pseudostates,
$\hat{I}_{2}^{\lr{N}}$, is of the form
\begin{equation}
  \label{eq:target-projection}
  \hat{I}_{2}^{\lr{N}}
  =
  \sum_{i = 1}^{N}
  \ket*{\phi_{i}^{\lr{N}}}
  \bra*{\phi_{i}^{\lr{N}}}
  =
  \sum_{i = 1}^{N_{B}}
  \ket*{\phi_{i}^{\lr{N}}}
  \bra*{\phi_{i}^{\lr{N}}}
  +
  \sum_{i = N_{B} + 1}^{N}
  \ket*{\phi_{i}^{\lr{N}}}
  \bra*{\phi_{i}^{\lr{N}}}
\end{equation}
and so in the limit as $N \to \infty$, the sum over the bounded pseudostates
will converge to the sum over the true target discrete states
\begin{equation}
  \label{eq:target-projection-discrete}
  \lim_{N \to \infty}
  \sum_{i = 1}^{N_{B}}
  \ket*{\phi_{i}^{\lr{N}}}
  \bra*{\phi_{i}^{\lr{N}}}
  =
  \sum_{i = 1}^{\infty}
  \ket*{\phi_{i}}
  \bra*{\phi_{i}}
\end{equation}
and the sum over the unbounded pseudostates will converge to a discretisation of
the integral over the true continuum spectrum
\begin{equation}
  \label{eq:target-projection-continuum}
  \lim_{N \to \infty}
  \sum_{i = N_{B} + 1}^{N}
  \ket*{\phi_{i}^{\lr{N}}}
  \bra*{\phi_{i}^{\lr{N}}}
  =
  \int_{\boldsymbol{q} : q^{2} \geq 0}
  \dd{\boldsymbol{q}}
  \ket*{\boldsymbol{q}}
  \bra*{\boldsymbol{q}}
  .
\end{equation}
Whence, it follows that projection operator for the target pseudostates
converges to the identity operator, for the space of target states,
in the limit as $N \to \infty$; that is,
\begin{equation}
  \label{eq:target-projection-convergence}
  \lim_{N \to \infty}
  I_{2}^{\lr{N}}
  =
  I_{2}
  .
\end{equation}

\paragraph{Total State}

The total state, $\ket*{\Psi^{\lr{S}}}$, with a given total spin $S$, are
eigenstates of the total Hamiltonian,
$\hat{H} = \hat{H}_{1} + \hat{H}_{2} + \hat{V}_{1, 2}$; that is,
\begin{equation}
  \label{eq:total-hamiltonian}
  \hat{H}
  \ket*{\Psi^{\lr{S}}}
  =
  E
  \ket*{\Psi^{\lr{S}}}
  .
\end{equation}
As a result of the asymptotic initial conditions, in which the projectile and
the target are assumed to be sufficiently far apart as to be non-interacting,
the asymptotic Hamiltonian, $\hat{H}_{\lr{+}} = \hat{K}_{1} + \hat{H}_{2}$,
satisfies
\begin{equation}
  \label{eq:total-asymptotic-hamiltonian}
  \hat{H}_{\lr{+}}
  \ket*{\phi_{i}^{\lr{N}}, \boldsymbol{k}}
  =
  \lr
  {
    \epsilon_{i}^{\lr{N}}
    +
    \tfrac{k^{2}}{2}
  }
  \ket*{\phi_{i}^{\lr{N}}, \boldsymbol{k}}
  .
\end{equation}
Naively, we might suppose that the total asymptotic state,
$\ket*{\Psi_{i}^{\lr{S, +}}}$, may be written in the form
$\ket*{\Psi_{i}^{\lr{S, +}}} = \ket{\phi_{i}, \boldsymbol{k}_{i}}$,
however, this does not account for the effects of spin and Pauli's exclusion
principle, or more generally symmetrisation / anti-symmetrisation.
For the case where the projectile is a fermion, and the target contains one such
fermion, then the fermions are necessarily indistinguishable.
Additionally, Pauli's exclusion principle states that two fermions cannot occupy
the same state (including spin) within a system.
This case is applicable for both e-H scattering and e-He scattering with a
frozen core model for He.
To account for this principle, we write the total state,
$\ket*{\Psi_{i}^{\lr{S, +}}}$, in the form
\begin{equation}
  \label{eq:total-state-symmetrisation}
  \ket*{\Psi_{i}^{\lr{S, +}}}
  =
  \lrsq
  {
    1
    +
    \lr{-1}^{S}
    \hat{P}_{r}
  }
  \ket*{\psi_{i}^{\lr{S, +}}}
\end{equation}
where $\hat{P}_{r}$ is the spatial exchange operator, and where
$\ket*{\psi_{i}^{\lr{S, +}}}$ is a two-fermion state, for which it follows that
the total state is guaranteed to possess the proper symmetry properties.
It then remains to determine the form of $\ket*{\psi_{i}^{\lr{S, +}}}$.
We note that by expanding this two-fermion state in the set of target states, we
have that
\begin{equation}
  \label{eq:target-expansion}
  \ket*{\psi_{i}^{\lr{S, +}}}
  =
  \hat{I}_{2}
  \ket*{\psi_{i}^{\lr{S, +}}}
  =
  \lim_{N \to \infty}
  \hat{I}_{2}^{\lr{N}}
  \ket*{\psi_{i}^{\lr{S, +}}}
  =
  \lim_{N \to \infty}
  \ket*{\psi_{i}^{\lr{N, S, +}}}
\end{equation}
where
\begin{equation}
  \label{eq:target-expansion-2}
  \ket*{\psi_{i}^{\lr{N, S, +}}}
  =
  \hat{I}_{2}^{\lr{N}}
  \ket*{\psi_{i}^{\lr{S, +}}}
  =
  \sum_{j = 1}^{N}
  \ket*{\phi_{j}^{\lr{N}}}
  \bra*{\phi_{j}^{\lr{N}}}
  \ket*{\psi_{i}^{\lr{S, +}}}
  .
\end{equation}

\subsubsection{Close-Coupling Equations}
\label{sec:cc-equations}

\subsubsection{Transition Amplitudes}
\label{sec:transition-amplitudes}

\subsubsection{Cross Sections}
\label{sec:cross-sections}

\paragraph{Total Cross Sections}

\paragraph{Differential Cross Sections}

\subsubsection{S-Wave Model}
\label{sec:s-wave-model}

\subsection{Electron-Impact Hydrogen Scattering}
\label{sec:e-h}

\subsubsection{Elastic Scattering}
\label{sec:e-h-elastic-scattering}

\subsubsection{Excitation}
\label{sec:e-h-excitation}

\subsubsection{Ionisation}
\label{sec:e-h-ionisation}

\paragraph{Singlet Case}

\paragraph{Triplet Case}

\subsection{Electron-Impact Helium Scattering}
\label{sec:e-he}

\subsubsection{Considerations for a Two-Electron Target}
\label{sec:2e-target}

\paragraph{Pauli Exclusion Principle}

\paragraph{Frozen-Core Model}

\subsubsection{Elastic Scattering}
\label{sec:e-he-elastic-scattering}

\subsubsection{Excitation}
\label{sec:e-he-excitation}

\paragraph{Auto-Ionisation}

\subsubsection{Ionisation}
\label{sec:e-he-ionisation}

\section{Survey of Experimental Literature}
\label{sec:survey-experimental}

\section{Survey of Theoretical Literature}
\label{sec:survey-theoretical}

\subsection{Electron-Impact Hydrogen Ionisation Calculations}
\label{sec:e-h-ionisation-calculations}

\subsubsection{Convergent Close-Coupling Calculations}
\label{sec:e-h-ccc-calculations}

\subsubsection{Exterior-Complex-Scaling Calculations}
\label{sec:e-h-ecs-calculations}

\subsubsection{Ansatz of Zatsarinny and Bartschat}
\label{sec:e-h-ecs-calculations}

\subsection{Electron-Impact Helium Ionisation Calculations}
\label{sec:e-he-ionisation-calculations}

\subsubsection{Convergent Close-Coupling Calculations}
\label{sec:e-he-ccc-calculations}

\subsubsection{Exterior-Complex-Scaling Calculations}
\label{sec:e-he-ecs-calculations}

\subsubsection{Ansatz of Zatsarinny and Bartschat}
\label{sec:e-he-ecs-calculations}

\section{Conclusion}
\label{sec:conclusion}

\clearpage

\bibliographystyle{chicago}

\bibliography{references}

\clearpage

\appendix

\section{Properties of Utilised Bases}
\label{app:properties}

\subsection{Spherical Harmonics}
\label{app:spherical-harmonics}

\subsubsection{Completeness}
\label{app:spherical-harmonic-completeness}

\subsection{Laguerre Radial Basis}
\label{app:laguerre-radial-basis}

\subsubsection{Completeness}
\label{app:laguerre-completeness}

\subsection{Laguerre Basis}
\label{app:laguerre-basis}

\subsection{Plane Waves}
\label{app:plane-waves}

\section{Numerical Techniques}
\label{app:numerical-techniques}

\subsection{Basis Expansion}
\label{app:basis-expansion}

\subsection{Diagonalisation}
\label{app:diagonalisation}

\end{document}